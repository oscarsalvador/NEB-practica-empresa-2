\newacronym{aws}{AWS}{Amazon Web Services}
\newacronym{s3}{S3}{Simple Storage Solution}
\newacronym{api}{API}{Application Programming Interface}
\newacronym{url}{URL}{Uniform Resource Locator}
\newacronym{npm}{NPM}{Node Package Manager}
\newacronym{rest}{REST}{REpresentational State Transfer}
\newacronym{saas}{SaaS}{Software as a Service}
\newacronym{bbdd}{BBDD}{Bases de Datos}
\newacronym{sspl}{SSPL}{Server-Side Public Licence}
\newacronym{mpl}{MPL}{Mozilla Public Licence}
\newacronym{gpl}{GPL}{General Public Licence}
\newacronym{agpl}{AGPL}{Affero General Public Licence}
\newacronym{aks}{AKS}{Azure Kubernetes Service}
\newacronym{aci}{ACI}{Azure Container Instance}
\newacronym{aca}{ACA}{Azure Container Apps}
\newacronym{fqdn}{FQDN}{Fully Qualified Domain Name}
\newacronym{cors}{CORS}{Cross-Origin Resource Sharing}
\newacronym{cli}{CLI}{Command Line Interface}
\newacronym{tui}{TUI}{Terminal User Interface}
\newacronym{iac}{IaC}{Infrastructure as Code}
\newacronym{crud}{CRUD}{Create, Read, Update, and Destroy}
\newacronym{tls}{TLS}{Transport Layer Security}
\newacronym{json}{JSON}{JavaScript Object Notation}
\newacronym{hcl}{HCL}{Hashicorp Configuration Language}
\newacronym{ip}{IP}{Internet Protocol}
\newacronym{cicd}{CI/CD}{Continuous Integration and Continuous Delivery (or Deployment, if completely automated)}
\newacronym{yaml}{YAML}{YAML Ain't Markup Language}
\newacronym{pr}{PR}{Pull Request}
\newacronym{ssh}{SSH}{Secure SHell}
\newacronym{arm}{ARM}{Azure Resource Manager}
\newacronym{cdk}{CDK}{Cloud Development Kit}
\newacronym{https}{HTTPS}{HyperText Transfer Protocol Secure}
\newacronym{acr}{ACR}{Azure Container Registry}
\newacronym{hadr}{HA/DR}{High Availability and Disaster Recovery}
\newacronym{csv}{CSV}{Comma Separated Vector}
\newacronym{corba}{CORBA}{Common Object Request Broker Arquitecture}
\newacronym{sdlc}{SDLC}{Software Development Life Cycle}
\newacronym{pid}{PID}{Process IDentifier}
\newacronym{ipc}{IPC}{InterProcess Communication}
\newacronym{vm}{VM}{Virtual Machine}
\newacronym{lxc}{LXC}{LinuX Containers}
\newacronym{oci}{OCI}{Open Container Initiative}
\newacronym{cncf}{CNCF}{Cloud Native Computing Foundation}
\newacronym{k8s}{K8s}{Kubernetes (K, 8 letters, s)}
\newacronym{dns}{DNS}{Domain Name Service}
\newacronym{cri}{CRI}{Kubernetes Container Runtime Interface}
\newacronym{crio}{CRI-O}{Container Runtime Interface - \\OpenShift}
\newacronym{pat}{PAT}{Personal Access Token}
\newacronym{uat}{UAT}{User Acceptance Testing}
\newacronym{ca}{CA}{Certificate Authority}
\newacronym{pki}{PKI}{Public Key Infrastructure}
\newacronym{csr}{CSR}{Certificate Signing Request}
\newacronym{owasp}{OWASP}{Open Worldwide Application Security Project}
\newacronym{cve}{CVE}{Common Vulnerabilities and Exposures}
\newacronym{xss}{XSS}{Cross-site Scripting}
\newacronym{sca}{SCA}{Software Composition Analysis}
\newacronym{sast}{SAST}{Static Application Security Testing}
\newacronym{dast}{DAST}{Dynamic Application Security Testing}
\newacronym{cva}{CVA}{Container Vulnerability Analysis}
\newacronym{acl}{ACL}{Access Control List}
\newacronym{ha}{HA}{High Availability}

\newglossaryentry{azure}{
    name={Azure},
    description={Microsoft's cloud platform}
}
\newglossaryentry{frontend}{
    name={Frontend},
    description={Server from which the user's browser downloads the files necessary to render the web page}
}
\newglossaryentry{client}{
    name={Client},
    description={Process that connects to a server, in the case of a web page, the code that runs in the user's browser}
}
\newglossaryentry{middleware}{
    name={Middleware},
    description={Intermediate server between the frontend and databases. Normally performs the Business Logic}
}
\newglossaryentry{backend}{
    name={Backend},
    description={Server that is not directly accessible to the user. For the purposes of this report, interchangeable with middleware}
}
\newglossaryentry{bucket}{
    name={Bucket},
    description={Storage unit in AWS}
}
\newglossaryentry{blob}{
    name={Blob},
    description={Storage unit in Azure Storage Contairnes}
}
\newglossaryentry{http_get}{
    name={HTTP GET},
    description={Método de HTTP en el que se pide un recurso}
}
\newglossaryentry{http_put}{
    name={HTTP PUT},
    description={Método de HTTP en el que se pide crear un recurso. Es idempotente}
}
\newglossaryentry{http_post}{
    name={HTTP POST},
    description={Método de HTTP en el que se pide subir datos. No es idempotente}
}
\newglossaryentry{token}{
    name={Token},
    description={También conocido como identificador de sesión, es una cadena de caracteres con la que seguir a un usuario, y darle acceso sin que tenga que enviar su contraseña repetidas veces}
}
\newglossaryentry{traefik}{
    name={Traefik},
    description={Proxy inverso de última generación, descubre servicios por su cuenta}
}
\newglossaryentry{reverse_proxy}{
    name={Reverse-Proxy},
    description={Proxy dentro del firewall, en particular usado por recursos dentro de la organización para acceder a otros también dentro. En mi caso, el valor es presentar una única dirección, saltándome el problema de CORS}
}
\newglossaryentry{wildcard}{
    name={Wildcard},
    description={Valor que representa una equivalencia a cualquier valor}
}
\newglossaryentry{infra}{
    name={Infrastructure},
    description={Set of hardware and software components that support an application}
}
\newglossaryentry{mongodb}{
    name={MongoDB},
    description={Non-SQL document database}
}
\newglossaryentry{redis}{
    name={Redis},
    description={Key-Value database that only stores in memory. Used as a cache, fast but non-persistent, for tokens}
}
\newglossaryentry{fullstack}{
    name={Fullstack},
    description={The totality of all systems, frontend, backend, and databases}
}
\newglossaryentry{docker}{
    name={Docker},
    description={Container-based virtualization software}
}
\newglossaryentry{graphql}{
    name={GraphQL},
    description={Graph Query Language, alternativa a REST, propuesta por Facebook}
}
\newglossaryentry{onprem}{
    name={On-Prem},
    description={``On-Premises'', Infrastructure on the premises of the company, instead of a third party's}
}
\newglossaryentry{wrapper}{
    name={Wrapper},
    description={Abstraction layer upon a given tool}
}
\newglossaryentry{cloud}{
    name={Cloud},
    description={Platform or service purchased as a service, a third party's computer and software that can be run on it}
}
\newglossaryentry{pipeline}{
    name={Pipeline},
    description={Set of automated processes, normally in a repository or cloud environment, that apply after being triggered by an event, such as a commit to said repo}
}
\newglossaryentry{overhead}{
    name={Overhead},
    description={Added cost incurred by management or supervisory processes}
}
\newglossaryentry{localhost}{
    name={Localhost},
    description={El nombre de host estándar para la mísma máquina}
}
\newglossaryentry{runtime}{
    name={Runtime},
    description={Period in the lifecycle of a program during which it is executed}
}
\newglossaryentry{registry}{
    name={Registry},
    description={Service to which container images can be uploaded}
}
\newglossaryentry{namespace}{
    name={Namespace},
    description={Mechanism for the isolation of groups of processes or resources}
}
\newglossaryentry{minikube}{
    name={Minikube},
    description={Local Kubernetes distribution made for prototyping}
}
\newglossaryentry{kubectl}{
    name={Kubectl},
    description={CLI to interact with a Kubernetes cluster's control plane through the API}
}
\newglossaryentry{helm}{
    name={Helm},
    description={Package (chart, configurations) manager for Kubernetes}
}
\newglossaryentry{adhoc}{
    name={Ad-Hoc},
    description={Specific to and end, without a previous plan or integration into a general solution}
}